\section{Hardware Architecture}

\begin{frame}
\frametitle{Hardware Architecture}
The hardware used in an autonomous vehicle project evolves with the different
stages of development.

\begin{itemize}
    \item Off-the-shelf hardware (x86 PCs, network switches, etc.) are used
        in early stages of development in order to focus on the functional
        software.
    \item Evaluation boards from chip vendors, or ruggedized version from
        third parties, are often used in pre-development stages to begin
        the work of optimizing the software for an embedded environment.
    \item Production hardware is produced in cooperation with a Tier 1
        supplier (e.g. Continental, ZF, Bosch, etc.), where several samples
        (A-sample to D-sample) are produced before the final version is made
        for start of production (SOP).
\end{itemize}
\end{frame}

\begin{frame}
\frametitle{Hardware Architecture}
\framesubtitle{Early prototyping}
Example of an early prototype with off-the-shelf hardware, but also prototype
hardware from automotive vendors (VIGEM, Elektrobit seen here).
\begin{center}
\includegraphics[width=0.5\textwidth]{images/bmw_vehicle_trunk.jpg}\\
\footnotesize Source: BMW\footnotemark[1]
\end{center}
\footnotetext[1]{\tiny{\url{https://www.press.bmwgroup.com/global/article/detail/T0320230EN/nextgen-2020}}}
\end{frame}

\begin{frame}
\frametitle{Hardware Architecture}
\framesubtitle{First embedded hardware}
Developer kits from chip vendors serve as a good platform for doing an initial
integration of a system to an embedded environment and optimizing the
application.
\begin{center}
\includegraphics[width=0.6\textwidth]{images/nvidia_drive_agx.jpg}\\
\footnotesize Source: NVIDIA\footnotemark[1]
\end{center}    
\footnotetext[1]{\tiny{\url{https://developer.nvidia.com/drive/drive-agx}}}
\end{frame}

\begin{frame}
\frametitle{Hardware Architecture}
\framesubtitle{Production hardware}
Tier 1 suppliers develop rugged and automotive grade ECUs for mass production,
often customized for an OEM's specific requirements.\\
\vspace{0.2cm}
\begin{columns}[]
    \begin{column}{0.5\textwidth}
        \centering
        \includegraphics[width=0.7\textwidth]{images/continental_ecu.jpg}\\
        \footnotesize Continental ADAS/AD ECU\footnotemark[1]
    \end{column}
    \begin{column}{0.5\textwidth}
        \centering
        \includegraphics[width=0.7\textwidth]{images/zf_ecu.jpg}\\
        \footnotesize ZF ProAI\footnotemark[2]
    \end{column}
\end{columns}
\footnotetext[1]{\tiny{\url{https://press.zf.com/press/en/releases/release_28483.html}}}
\footnotetext[2]{\tiny{\url{https://www.continental-automotive.com/en-gl/Passenger-Cars/Autonomous-Mobility/Enablers/Control-Units/Assisted-Automated-Driving-Control-Unit}}}
\end{frame}

\begin{frame}
\frametitle{Hardware Architecture}
\framesubtitle{Designing for safety and performance}
Talk about different type of cores required in a final hardware architecture,
e.g. microcontrollers, safety cores, hardware acceleration, security, etc.

Mentioned that software should abstract all of this away --> but this is a big
challenger (mentioned what Apex.AI is doing to solve this)
\end{frame}